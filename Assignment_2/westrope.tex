\documentclass[12pt]{article}
\author{Ian Westrope}
\title{Evolutionary Computation Homework Two}
\date{\today}

\begin{document}

\maketitle

\section{Report}
This experiment was to play around with simulated annealing, by controlling the mutation, starting temperature, and cooling constant. What I found was that the Cauchy distribution mutation did the best no matter what the temperature was, while the normal and uniform mutations did about the same. Out of the tests ran the best starting temperature was 100, and the best cooling constant 0.99. But these numbers are not optimal. When testing on a function with only one maximum I got much better results with a temperature of 50 and a cooling rate of 0.5. Where the other numbers gave mixed results not always finding the maximum. 




\end{document}


